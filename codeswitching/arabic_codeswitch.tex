\documentclass[10pt, a4paper]{article}
\usepackage{lrec2006}
\usepackage{graphicx}
%\usepackage{natbib}
\usepackage{float}
\usepackage{arabtex}
\usepackage{utf8}

\title{An Algerian Arabic-French Code-Switched Corpus}

\name{Ryan Cotterell$^1$, Adithya Renduchintala$^1$, Naomi Saphra$^1$, Chris Callison-Burch$^2$}
\address{ 
 $^1$ Center for Language and Speech Processing, Johns Hopkins University \\
 $^2$ Computer and Information Science Department, University of Pennsylvania \\}
%              \{ryan.cotterell, adithya.renduchintala,nsaphra\}@jhu.edu; ccb@cis.upenn.edu}

\abstract{ Arabic is not just one language, but rather a collection of
dialects in addition to Modern Standard Arabic (MSA). While MSA is
used in formal situations, dialects are the language of every day
life. Until recently, there was very little dialectal Arabic in
written form. With the advent of social-media, however, the landscape
has changed.  We provide the first romanized code-switched Algerian
Arabic-French corpus annotated for word-level language id. We review
the history and sociological factors that make the linguistic
situation in Algerian unique and highlight the value of this corpus to
the natural language processing and linguistics communities. To build
this corpus, we crawled an Algerian newspaper and extracted the
comments from the news story. We discuss the informal nature of the
language in the corpus and the challenges it will
present. Additionally, we provide a preliminary analysis of the
corpus. We then discuss some potential uses of our corpus of interest
to the computational linguistics community. \newline
\\\Keywords{code-switching, Algerian Arabic, romanized Arabic, French}}


\begin{document}
\setcode{utf8}


\maketitleabstract


\section{Introduction}
Language identification systems have long operated under the
assumption that text is written in a single language. As social media
becomes a more prominent mode of communication, such systems are
confronting text that increasingly challenges the monolingual
assumption. More than half the world's population is bilingual and
information communication is often code-switched, reflecting the need
for a deeper understanding of code-switching in relation to NLP
tasks. Recent work has proposed both supervised and unsupervised
methods for \textit{word}-level language id. Current methods, however,
rely on the assumption that external resources exist, such as large
non-code-switched corpora and dictionaries. These resources are not
available for some languages and dialects, including Algerian vernacular
Arabic, a dialect often code-switched with French. We are releasing a
corpus of romanized Algerian Arabic and French scraped from the
comments sections of Echorouk, an Algerian daily newspaper with the
second-largest reader base of any Arabic paper. This is the only
significant corpus of romanized Arabic known to the authors and
additionally it is the largest corpus of code-switched data to our
knowledge. Such a resource is necessary because romanized Arabic is
becoming increasingly popular on the internet. The corpus will also be
of use for the linguistic study of code-switching. Much of previous
code-switching research has focused on data collected from field work,
and a found dataset like ours could provide an interesting perspective
on the use of code-switching in conversation.

\section{Code-switching}

Code-switching is a linguistic phenomenon wherein speakers switch
between two or more languages in conversation, often within a single
utterance \cite{bullock2009cambridge}.  It can be viewed through a
sociolinguistic lens where situation and topic influence the
choice of language \cite{kachru1977linguistic}. To define code-switching as a phenomenon, it is
important to make the distinction between \emph{code-switching} and
\emph{borrowing}%, which has implications for annotation and will be
%discussed later
. \emph{Borrowing} is the act of using a foreign word
without recourse to syntactic or morphological properties of that
language and often occurs with phonological
assimilation. \emph{Code-switching}, on the other hand, involves
switching between languages in which the speakers are fluent, and can
in effect be viewed as changing the grammar in use. Some linguists
have even proposed a scale of code-switching, positing the existence
of a continuum between \emph{borrowing} and \emph{code switching} \cite{auer1999codeswitching}.
Code-switching points (the times at which speakers change language)
and the context around which switching occurs are also of interest to
linguists. These points often lie within a sentence and their position
is influenced by the syntax of the respective languages. Poplack
\cite{poplack1988contrasting} posited that code-switching points
cannot occur within a constituent. Recent work, however, has found
that many speakers relax this constraint. The Matrix Language-Frame
(MLF) model is one theory has gained traction \cite{myers1993common}
to explain code-switching patterns.  MLF proposes that there is a
Matrix Language (ML) and an Embedded Language (EL). The ML is the more
dominant language and is often the language which the speaker
identifies as their native tongue. The EL is then inserted into the ML
at certain grammatical frames. Within this framework, further work has
gone into the exact syntactic and morphological contexts that allow
for code-switching points \cite{myers2001calculating}.

\section{Code-Switching in North Africa}
Code-switching in North African Arabic is an established phenomenon
that has been studied by the linguistics community
\cite{bentahila1983syntax}. It dates back to the initial French
colonization of North Africa. North Africa is also home to many
cultures, a fact which potentially affects language use and code
switching in particular. Until recently, mixed language communication
has been observed mainly as a spoken phenomenon. With the widespread
use of computer-mediated communication, code-switching is becoming
common in North African Arabic \cite{Salia-11}. Comments on news-feeds
and social media outlets like Twitter and Facebook often contain
code-switching.  North African Arabic is not the only language to
appear in code-switching writing. A body of recent sociolinguistics
work has considered the phenomenon in various settings. Swiss-German
and German code-switching in chat rooms was analyzed in
\newcite{siebenhaar2006code} and \newcite{callahan2004spanish}
considered Spanish and English code-switching in a written corpus.

\section{Related Work}
From a computational perspective, code-switching has received
relatively little attention. \newcite{joshi1982processing} provides
a tool for parsing mixed sentences. More recently,
\newcite{rosner2007tagging} focused on processing code-switched
SMS. \newcite{solorio2008learning} trained classifiers to predict
code-switching points in Spanish and English. \newcite{nguyen2013word}
also focused on word level language identification in Dutch-Turkish
news commentary. To our knowledge, \newcite{elfardy2012token} is the
only computational work on Arabic code-switching done to date. That
work does not include romanized Arabic. Many languages written in a
non-Roman script are \emph{romanized} on the internet. This practice
presents a problem for standard NLP tools that are trained on the
language with its standard orthography \cite{irvine2012processing}. We
believe that this type of romanized data will become more pervasive as
more users employ computer-mediated communication globally. More information
will be generated in such settings, and it is critical for future NLP
systems to be able to process the data produced. The corpus we are
presenting is a step in this direction.

\section{Data Collection}
We used a corpus crawled from an Algerian newspaper
website. We scraped 598047 pages in September 2012. These fora are
rich in both dialectal Arabic and French content. The
corpus contains discussion on a wide-ranging set of issues including
domestic politics, international relations, religion, and sporting
events.  We extracted 6949 comments, containing 150,000 words in
total. We separated the comments section from the main article on each
page and stripped HTML tags and other non-user generated content. The
metadata was stripped in an attempt to preserve anonymity.  The Arabic
portion of the corpus was annotated for sentence level dialect on
Mechanical Turk \cite{cotterell2014extended}.

We separated all the comments, in which more than half the non-white
space characters were in the Roman alphabet, determining these to be
romanized.  We did no further processing, e.g. tokenization.  The
final data set contains 339504 comments with an average
length of 19 tokens, as determined by separating on white space and
punctuation. 1000 of the comments are annotated using
the guidelines described below. Our corpus has 493038 types and 6718502 tokens, and is
formatted in JSON.
\section{Romanization of Arabic}
This corpus is unique in that it is the first large corpus to the
authors' knowledge that is composed of of an Arabic dialect written in
romanized form. Romanized Arabic is particularly difficult because
there is no standard form of romanization used across the Arab
world. In order to use standard NLP tools on such corpora, it is often
necessary to \emph{deromanize} the corpus. In the case of Urdu, this
task has been successfully completed using standard Machine
Translation software \cite{irvine2012processing}.

Arabic written in the Latin alphabet, often dubbed \emph{arabizi}, is
extremely common on the internet and SMS.  The exact mapping from the
Arabic script onto the Latin alphabet varies significantly between
regions.  The specific case of romanization by young speakers of Gulf
Arabic in the United Arab Emirates is discussed thoroughly in
\newcite{palfreyman2003funky}.  

Figure \ref{fig:correspondence} expresses the most common mappings across the Arab
world.  Algerians, and North Africans in general, tend to use
romanizations that reflect French orthography: for instance \RL{ و}
$\mapsto$ \texttt{ou}, \RL{ش} $\mapsto$ \texttt{ch}, \RL{ج} $\mapsto$
\texttt{dj} and \RL{ا} $\mapsto$ \texttt{\`{e}} or \texttt{\'{e}}. 
%We
%illustrate this difference in figures 2 and 3.  In figure 2 we show
%the frequency of the common transcriptions of \RL{إن شاء الله} (God
%willing) with \RL{ش} $\mapsto$ \texttt{ch} and in figure 3 we show the
%transcriptions with \RL{ش} $\mapsto$ \texttt{sh}. We see that the ones
%in our corpus occur with \texttt{ch} occur about an order of magnitude
%more often.
To illustrate this difference consider the frequency of the common 
transcriptions of \RL{إن شاء الله} (God willing); 
we see RL{ش} $\mapsto$ \texttt{ch} about an order of magnitude
more often than \RL{ش} $\mapsto$ \texttt{sh}.

This transcription variation makes it unlikely that a single,
general-purpose Arabic deromanization tool will be enough, and such
romanized corpora will need to be developed for other dialects as well
in order to analyze the users romanization preferences on dialectal
basis.

\begin{figure}
\centering
\begin{tabular}{| l || r ||| l || r |  }
  \hline                  
  \textbf{Arabic} & \textbf{Arabizi} & \textbf{Arabic} & \textbf{Arabizi} \\ \hline  
  \RL{ا} &  a  & \RL{ب}  & b, p    \\
  \RL{ ت}  & t  & \RL{ ث}  &  th, s  \\
  \RL{ ج }  & j, g  & \RL{ ح }  & 7, h \\
  \RL{ خ }  &  7', 5  &\RL{ د }  & d  \\
  \RL{ ذ }  & th z & \RL{ ر }  & r   \\
  \RL{ ز }  & z  & \RL{ س }  & s  c \\
  \RL{ ش }  & sh, ch  & \RL{ ص }  &  9  \\ 
  \RL{ ض }  & 9', d  & \RL{ ط  }  & t   \\
  \RL{ ظ  }  & th &  \RL{ ع }  & 3  \\
  \RL{ غ  }  & gh, 3'  & \RL{ ف  }  & f   \\
  \RL{ ق} & 8, 2, k, q & \RL{ ك  }  & k \\
  \RL{ ل  }  & l  & \RL{ م }  & m  \\
  \RL{ ن }  & n  & \RL{ ه  }  & h \\
  \RL{ و  }  & w, o, ou & \RL{ ي  }  & y, i, e \\
  \hline  
\end{tabular}
\caption{Correspondence Between Arabic Letters and Romanized Arabic \cite{yaghan2008arabizi}}
\label{fig:correspondence}
\end{figure}


\section{Text Analysis}
Because our Algerian corpus is from the length-constrained informal
domain of online forum comments, it would be difficult to process
meaningfully without normalizing beforehand. It exhibits extreme
variation in spelling and grammar. Many forms of the same word may
appear throughout our corpus. For example, we identified 69 variants
of the common word \RL{إن شاء الله} alone. 70\% of all token types in
our corpus appeared only once, so the OOV rate in this forum corpus
can confound language processing systems without text normalization.
 Several typical sources of variation for Arabic identified by
\cite{Darwish:2012:LPA:2396761.2398658} were found in our corpus.
\begin{itemize}
\item The use of elongations, especially in the form of vowel repetition.
\begin{quote}
we ki ta3arfou wach rah testfadou ? {\bf hhhhh} cha3ab {\bf kar3adjiiiiiiiiii}
\end{quote}

\item Spelling mistakes, such as dropped or transposed characters.

\item Abbreviations.

\begin{quote}
rah thablona bel {\bf BAC} had al3am !!!!!
\end{quote}

\item Emotional tokens and ejaculative abbreviations, such as the abbreviation ``lol'' borrowed from
  English web speech, or emoticons.
\end{itemize}

In addition to these irregularities, our corpus contains variations
particular to romanized Arabic text because there is no standardized way
to transcribe Arabic orthography in this informal domain, Arabic words can be represented by multiple spellings. 
%Fortunately, knowing that the dialect of Arabic represented here is Algerian, we can predict likely transcriptions of characters based on the preponderence of French in
%Algerian language. 
%For example, as stated previously,  we may encounter both \emph{sh} and
%\emph{ch} as transciption variants of the same character, but in Algerian, French transcriptions dominate. Tables TODO and TODO demonstrate the distribution of the most common spelling
%variants for \RL{إن شاء
%  الله}. As we see, for these common variants the French transcription
%\emph{ch} appears almost 40 times as often as \emph{sh}.

% \begin{figure}
% 		\caption{Frequency of spellings with \emph{sh} transcriptions}
% 		\centering
% 		\includegraphics[scale=0.5]{sh-spelling.png}
% \end{figure}
% \begin{figure}
% \caption{Frequency of spellings with \emph{ch} transcriptions}
% \centering
% \includegraphics[scale=.5]{ch-spelling.png}
% \end{figure}
		

\section{Example Posts} We present below a few example sentences that
we have collected use the data collection methodology described above.
\begin{itemize}
  \item bezaf m3a saifi oalah mnkalifoha mairbahch\\
  {\it Had enough with Saifi.}
   \item la howla wa la kowata il bi lah el3alier l3adim wa la
yassa3oni an akoul anaho kllo chaye momkin ma3a ljazairyine\\ 
{\it For Gods Sake! I can just say that anything is possible with Algerians .}
   \item we ki ta3arfou wach rah testfadou ? hhhhh cha3ab
kar3adjiiiiiiiiii\\ 
{\it Don't try to know everything because it does not matter to you.}
   \item 7ade said nchalah li lmontakhabina el3assekari nchalah yjibo
natija mli7a bitawfiiiiiiiiiiiiiiiiik \ldots onchoriya chorouk
stp\\
{\it Good luck to our military team, I hope they get a good
score. Good Luck! Say it chorouk!}
   \item ya khawti tt simplement c bajiou l3arab o makanech fi tzayer
kamla joueur kima lhadji et on vai ras le 27 03 2011 chkoun houma rjal
liyestahlouha\\
{\it My brothers he's simply the Baggio of Arabs, and there
is none like Lhadji in Algeria, and on 27th of march 2011 we will see
who wins.}
   \item mais les filles ta3na ysedkou n'import quoi ana hada face
book jamais cheftou khlah kalbi\\
{\it Our girls believe anything, I have never seen this Facebook before.}
\end{itemize}

\section{Annotation Guidelines}
Annotating for word level language identification in code-switched
text is a difficult task because whether a word is code-switched is
often more of a continuum than a binary decision. Place names form a
simple example: \RL{باريس} (Paris) is an MSA word in that it is found
in most Arabic dictionaries, but it is clearly of French origin.  On
the other hand, \RL{فيديو} (video) is an example of a recent borrowing
from European languages that should be considered an Arabic word. To
simplify the decision, we made use of guidelines for dialectal Arabic
annotation provided in \cite{elfardy2012token}.  Their guidelines were
created for annotating world level language id in a corpus composed of
mixed dialectal Arabic and MSA both written in the Arabic script. As
we annotating two linguistically dissimilar languages, the same level
of ambiguity does not arise.

Further research in area of code-switching should focus on richer
annotation schemata that are both linguistically motivated,
i.e. taking into account the continuum of code-switching, and serve
practical NLP needs. Another interesting area to focus on could be to
annotate broad categories describing the type of
code-switching. \newcite{Kecskes-02} describes three prominent
patterns in code-switching (Insertion, Alternation and Congruent
Lexicalization) based on Gibraltar data. Insertion involves adding
lexical items from one language into the structure of the
other. Alternation is similar to insertion except that larger chunks
are inserted, rather than single tokens. Congruent Lexicalization is
adding lexical items from different lexical inventories into a common
grammar structure. The dataset could annotate each point of
code-switching with these patterns of code-switching as well.
 This data set, however, is focused on only the
points of code-switching.

The annotators were presented with posts and asked to label each word,
split on white space and punctuation.  They were given the choice of
Arabic (A), French (F) and Other (O). We excluded punctuation from the
annotation. Figure 4 shows the distribution of these tags in our
dataset. The annotation was conducted with an interactive Python
script.
% \begin{figure}
% \caption{Proportion of Arabic, French and Other tokens} % \centering
% \includegraphics[scale=.5]{pi-tags.png}
% \end{figure}

\section{Potential Uses}
The corpus provided is the first of its kind in that it is the first
large corpus of romanized Arabic that is code-switched with another
language. This has numerous potential uses in both the NLP community
and the linguistics community. In the NLP community, the processing of
informal text is becoming and increasingly popular task among
researchers \cite{yang2013log}. This corpus adds another complication
to informal text processing with the addition of code-switching. In
the linguistics community, a corpus based analysis of a code-switched
corpus offers the possibility to test various hypotheses on large
number of documents. The MLF hypothesis has already been studied in
bilingual speech corpora from Miami, Patagonia, and Wales (Carter,
2010), but it will be necessary to study such theories in the context
of many distinct languages and cultures to gain deeper insight into
the code-switching phenomenon.


\section{Conclusion and Future Work}
The primary contribution of this paper is the release of a Algerian
Arabic-French code-switched corpus. We have made use of a previously
proposed annotation scheme for word level language identification and
highlighted the unusual qualities of this corpus that make it a
significant contribution to field. Future work in this line should
largely focus on experiments using the corpus both in NLP and
linguistics. It would also be of interest to construct and annotate
similar corpora for other informal code-switched Arabic dialects.


\section*{Acknowledgements}
This material is based on research sponsored by a DARPA Computer Science Study Panel phase 3 award entitled ``Crowdsourcing Translation'' (contract D12PC00368). The views and conclusions contained in this publication are those of the authors and should not be interpreted as representing official policies or endorsements by DARPA or the U.S. Government. This research was supported by the Johns Hopkins University Human Language Technology Center of Excellence.



\bibliographystyle{lrec2006}
\bibliography{arabic_codeswitch}

\end{document}

